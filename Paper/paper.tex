% This is samplepaper.tex, a sample chapter demonstrating the
% LLNCS macro package for Springer Computer Science proceedings;
% Version 2.20 of 2017/10/04
%
\documentclass[runningheads]{llncs}
%
\usepackage{graphicx}
% Used for displaying a sample figure. If possible, figure files should
% be included in EPS format.
%
% If you use the hyperref package, please uncomment the following line
% to display URLs in blue roman font according to Springer's eBook style:
% \renewcommand\UrlFont{\color{blue}\rmfamily}

\begin{document}
%
\title{Subtle Gestures with the J!NS MEME}
%
%\titlerunning{Abbreviated paper title}
% If the paper title is too long for the running head, you can set
% an abbreviated paper title here
%
\author{Thomas Bartel \and
Florian Bossert \and
Sajjad Ahmad}
%
% TODO should we use 'et al.' here? ~FB
\authorrunning{T. Bartel \and F. Bossert \and S. Ahmad}
% First names are abbreviated in the running head.
% If there are more than two authors, 'et al.' is used.
%
\institute{Karlsruhe Institute of Technology, Karlsruhe, Germany}
%
\maketitle              % typeset the header of the contribution
%
\begin{abstract}
The abstract should briefly summarize the contents of the paper in
15--250 words.

\keywords{First keyword  \and Second keyword \and Another keyword.}
\end{abstract}
%
%
%
\section{Introduction}
\subsection{A Subsection Sample}
Please note that the first paragraph of a section or subsection is
not indented. The first paragraph that follows a table, figure,
equation etc. does not need an indent, either.

Subsequent paragraphs, however, are indented.

% As a reference, maybe useful later
Current smartglasses:
Smartglasses with cameras: Spectacles by Snap Inc. https://www.spectacles.com/
Smartglasses with EOG sensors: J!ns Meme by J!ns (website currently only in Japanese): https://jins-meme.com/

% Assigned to FB
\section{Related Work}
Our work touches on multiple areas of research, including wearable sensing, gesture recognition,
and subtle interaction.

\subsection{Face Interaction}
Detecting finger tapping on the body using acoustic sensing \cite{10.1145/1753326.1753394}.
Optical tracking system with infrared markers on fingers \cite{10.1145/2556288.2556984}.
Exploring touch interaction by placing electrodes around the ear \cite{10.1145/2468356.2468592}.
Eyelid gestures \cite{10.5555/2788890.2788938}.
Finger sensing IMUs \cite{10.1145/2858036.2858466}.
Hand sensing IMUs \cite{10.1145/2984511.2984582}.
Gesture recognition using wrist band \cite{10.1145/3274783.3274854}.
% Applications for Wearable Sensing
Tracking eating episodes \cite{10.1145/3130902}.
Sensors in shoes for tracking exercises \cite{10.1145/3174910.3174938}.
Using hand gestures to control TV with wrist band \cite{10.1145/2641248.2641359}.
Jump performance analysis \cite{10.1145/2753509.2753512}.
Monitoring drinking with wrist-worn inertial sensors \cite{10.1145/3267242.3267253}.

\subsection{EOG Sensing and Subtle Interaction}
While the aforementioned work utilized mainly IMUs, touch sensors or optical systems,
Electrooculography (EOG) has become a useful tool for eye-tracking in recent years, too. 
In 2006, Manabe et al. attached EOG sensors to over-ear \cite{10.1145/1125451.1125655}
and in 2013, to in-ear head phones \cite{10.1145/2493988.2494329} to detect eye gestures.
Bulling et al. first demonstrated efficient real-time eye-movement recognition
with EOG goggles in 2009, which they custom-made \cite{10.1145/1520340.1520468}.
Once commercial EOG glasses became available in 2014, Ichimaru et al. used them for
activity recognition \cite{10.1145/2638728.2638795}. They used the J!ns Meme
smartglasses to distinguish between typing, reading, eating and talking.
As for more recent work, Li et al. used the J!ns Meme to infer the physical and social
context of the wearer by recognizing various greeting gestures involving kissing
\cite{10.1145/3384657.3384801}.

In contrast to the work discussed, we focused on subtle interaction. Pohl et al. define
four types of subtle interaction: (1) signifying feedback that is non-intrusive to the
user, (2) hiding interaction from others and potentially deceiving them, (3) employing
less effort for input and generally doing less, and (4) nudging users
\cite{10.1145/3290605.3300648}. Our work aims at type (2) social subtlety, like e.g.
Ashbrook et al. did with their tooth-click gesture recognition interface
\cite{10.1145/2935334.2935389}. With a similar goal, Li et al. developed TongueBoard,
a retainer form-factor device for recognizing silent speech \cite{10.1145/3311823.3311831}.

The paper \textit{Itchy Nose} by Lee et al. is the most closely related work to ours,
as it inspired our efforts \cite{10.1145/3123021.3123060}. They used the J!ns Meme
smartglasses to detect multiple gestures that consisted of rubbing or pushing their
noses, which can also be detected by the glasses' EOG sensors.

% Assigned to FB
\section{Analysis}
Explaining the J!ns Meme.

What was our workflow?

Which gestures did we choose?

How does the J!ns Meme work and how do we use it?

What data do we collect, how and why?
Itchy Nose tool for automating personalization \cite{10.1145/3174910.3174953}.

% Assigned to TB
\section{Implementation}
\subsection{A Subsection Sample}
Please note that the first paragraph of a section or subsection is
not indented. The first paragraph that follows a table, figure,
equation etc. does not need an indent, either.

Subsequent paragraphs, however, are indented.

% Assinged to SA
\section{Results}
\subsection{A Subsection Sample}
Please note that the first paragraph of a section or subsection is
not indented. The first paragraph that follows a table, figure,
equation etc. does not need an indent, either.

Subsequent paragraphs, however, are indented.

% Assigned to TB
\section{Discussion}
\subsection{A Subsection Sample}
Please note that the first paragraph of a section or subsection is
not indented. The first paragraph that follows a table, figure,
equation etc. does not need an indent, either.

Subsequent paragraphs, however, are indented.

\section{Future Work}
\subsection{A Subsection Sample}
Please note that the first paragraph of a section or subsection is
not indented. The first paragraph that follows a table, figure,
equation etc. does not need an indent, either.

Subsequent paragraphs, however, are indented.

\section{First Section}
\subsection{A Subsection Sample}
Please note that the first paragraph of a section or subsection is
not indented. The first paragraph that follows a table, figure,
equation etc. does not need an indent, either.

Subsequent paragraphs, however, are indented.

\subsubsection{Sample Heading (Third Level)} Only two levels of
headings should be numbered. Lower level headings remain unnumbered;
they are formatted as run-in headings.

\paragraph{Sample Heading (Fourth Level)}
The contribution should contain no more than four levels of
headings. Table~\ref{tab1} gives a summary of all heading levels.

\begin{table}
\caption{Table captions should be placed above the
tables.}\label{tab1}
\begin{tabular}{|l|l|l|}
\hline
Heading level &  Example & Font size and style\\
\hline
Title (centered) &  {\Large\bfseries Lecture Notes} & 14 point, bold\\
1st-level heading &  {\large\bfseries 1 Introduction} & 12 point, bold\\
2nd-level heading & {\bfseries 2.1 Printing Area} & 10 point, bold\\
3rd-level heading & {\bfseries Run-in Heading in Bold.} Text follows & 10 point, bold\\
4th-level heading & {\itshape Lowest Level Heading.} Text follows & 10 point, italic\\
\hline
\end{tabular}
\end{table}


\noindent Displayed equations are centered and set on a separate
line.
\begin{equation}
x + y = z
\end{equation}
Please try to avoid rasterized images for line-art diagrams and
schemas. Whenever possible, use vector graphics instead (see
Fig.~\ref{fig1}).

\begin{figure}
\includegraphics[width=\textwidth]{fig1.eps}
\caption{A figure caption is always placed below the illustration.
Please note that short captions are centered, while long ones are
justified by the macro package automatically.} \label{fig1}
\end{figure}

\begin{theorem}
This is a sample theorem. The run-in heading is set in bold, while
the following text appears in italics. Definitions, lemmas,
propositions, and corollaries are styled the same way.
\end{theorem}
%
% the environments 'definition', 'lemma', 'proposition', 'corollary',
% 'remark', and 'example' are defined in the LLNCS documentclass as well.
%
\begin{proof}
Proofs, examples, and remarks have the initial word in italics,
while the following text appears in normal font.
\end{proof}
For citations of references, we prefer the use of square brackets
and consecutive numbers. Citations using labels or the author/year
convention are also acceptable. The following bibliography provides
a sample reference list with entries for journal
% articles~\cite{ref_article1}, an LNCS chapter~\cite{ref_lncs1}, a
% book~\cite{ref_book1}, proceedings without editors~\cite{ref_proc1},
% and a homepage~\cite{ref_url1}. Multiple citations are grouped
% \cite{ref_article1,ref_lncs1,ref_book1},
% \cite{ref_article1,ref_book1,ref_proc1,ref_url1}.
%
% ---- Bibliography ----
%
% BibTeX users should specify bibliography style 'splncs04'.
% References will then be sorted and formatted in the correct style.
%
\bibliographystyle{splncs04}
\bibliography{mybibliography}
%
\end{document}
